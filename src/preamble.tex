% ================================================================
% LOAD PACKAGES THAT ONLY NEED DEFAULT OPTIONS

% Default AMS packages
\usepackage{
  amsmath,   % Extended equation environments
  amssymb,   % Extended maths symbols (also loads amsfonts)
  amsthm,    % Theorem-like structures
} 

% Graphics packages
\usepackage{
  caption,     % Write captions 
  graphicx,    % Include graphics (with extras)
  pgfplots,    % For charts
  tikz,        % Draw tikz pictures
  subcaption,  % Write subcaptions
}
\pgfplotsset{compat=1.13}  % set version

% Formatting packages
\usepackage{
  datetime,    % Output date and time 
  fancyhdr,    % Extended header and footer options 
  hyperref,    % Generate hyperlinks for references
  multicol,    % Multicolumn environment
  natbib,      % Better inline citations (\citep, \citet, \citeauthor)
}


% ================================================================
% LOAD PACKAGES THAT NEED CONFIG OPTIONS AND USER PACKAGES

% required for \source command
\usepackage[
  absolute,
  overlay
]{textpos}  


% ================================================================
% DECLARE MATHS MACROS 

% Common symbols
\def\vep{\varepsilon}           % \vep - Script eps
\def\del{\partial}              % \del Partial derivative
\def\P{\mathbb{P}}              % \P - Probability measure
\def\E{\mathbb{E}}              % \E - Expectation operator
\def\Var{\text{Var}}            % \Var - Variance operator
\def\Cov{\text{Cov}}            % \Cov - Covariance
\def\F{\mathcal{F}}             % \F - Sigma-field
\def\Tr{\text{Tr}}              % \Tr - Matrix trace
\def\Det{\text{Det}}            % \Det - Matrix determinant
\def\grad{\boldsymbol{\nabla}}  % \grad - Gradient operator
\def\lap{\nabla^2}              % \lap - Laplacian
\def\I{\mathbb{I}}              % \I - Indicator 

% Common structures
\newcommand{\br}[1]{\left \lbrace #1 \right \rbrace}     % \br{.} - braces
\newcommand{\abs}[1]{\left \vert #1 \right \vert}        % \abs{.} - abs val
\newcommand{\floor}[1]{\left \lfloor #1 \right \rfloor}  % \floor{.} - floor
\newcommand{\ceil}[1]{\left \lceil #1 \right \rceil}     % \ceil{.} - ceiling 
\newcommand{\norm}[1]{\left \|#1 \right \|}              % \norm{.} - norm
\newcommand{\vc}[1]{\boldsymbol{#1}}                     % \vc{.} - vector (bb)

\newcommand{\ncr}[2]{  % \ncr{}{} - Combination
  \left(               % e.g. \ncr{25}{3}
    \begin{matrix} 
      #1 \\ #2 
    \end{matrix} 
  \right)
}

\newcommand{\ddx}[3][]{  % \ddx[]{}{} - Derivative operator 
  \frac{d^{#1} {#2}}     % e.g. \ddx{f}{x}, \ddx[2]{f}{x}
    {d{#3}^{#1}}  
} 

\newcommand{\ppx}[3][]{      % \ppx[]{}{} - Partial derivative operator 
  \frac{\partial^{#1} {#2}}  % \ppx{f}{x}, \ppx[2]{f}{x}
    {\partial{#3}^{#1}}      % TODO: update command for mixed partials
}


% ================================================================
% DOCUMENT FORMATTING

% Define \source{} to place references in bottom left of slides
\newcommand\source[1]{
  \begin{textblock*}{0.66\paperwidth}[0,1](6pt,1.055\textheight)
    \usebeamerfont{framesource}
    \usebeamercolor[fg]{framesource} 
    \fontsize{6pt}{7.2}
    \selectfont 
    \raggedright 
    Source: #1
    \hspace{.5em}
  \end{textblock*}
}

% Define commands to switch navigation bullets on and off
% Source: https://tex.stackexchange.com/a/45038
\makeatletter
  \let\beamer@writeslidentry@navbulon
    =\beamer@writeslidentry

  \def\beamer@writeslidentry@navbuloff{
    \expandafter\beamer@ifempty\expandafter{
      \beamer@framestartpage
    }{}% does not happen normally
    {%else
      % removed \addtocontents commands
      \clearpage
      \beamer@notesactions
    }
  }

  \newcommand*{\navbulon}{
    \let\beamer@writeslidentry=
      \beamer@writeslidentry@navbulon
  }

  \newcommand*{\navbuloff}{
    \let\beamer@writeslidentry=
      \beamer@writeslidentry@navbuloff
  }
\makeatother


% Select beamer theme
\usetheme{Frankfurt}

% Select beamer colours
\definecolor{UOMblue}{RGB}{0,69,124}
\usecolortheme[named=UOMblue]{structure}

% Define date format for final
\newdateformat{monthyeardate}{
  \monthname[\THEMONTH], \THEYEAR
}

% Define date format for draft
\newdateformat{draftdate}{
  DRAFT \\
  \currenttime 
  \ifx\TIMEZONE\undefined
  \else
    \space \TIMEZONE 
  \fi
  \space - \space 
  \ordinal{DAY} of \monthname[\THEMONTH], 
  \THEYEAR
}

% Write outline of section slide on each new section
\AtBeginSection[]
{
  \navbuloff
	\begin{frame}<beamer>{Outline}
		\tableofcontents[currentsection,hideothersubsections]
  \end{frame}
  \navbulon
}

% Clear section title for references
\renewcommand\bibsection{}

